\documentclass{article}
\usepackage[utf8]{inputenc}
\usepackage{dsfont}
\usepackage[spanish]{babel}
\usepackage[a4paper, total={6in, 8.5in}]{geometry}
\usepackage{multicol}
\usepackage{multicol}
\usepackage{amsmath}
\usepackage{amssymb}
\usepackage{amsthm}
\usepackage{physics}
\usepackage{calrsfs}
\usepackage{mathrsfs}
\usepackage{calligra}

 \usepackage{layout}

 \theoremstyle{plain}
 \newtheorem{thm}{Teorema}
 \newtheorem{prop}{Proposición}
 \newtheorem{lema}{Lema}
 \newtheorem{coro}{Corolario}

 \theoremstyle{definition}
 \newtheorem{defn}{Definición}
 \newtheorem{exmp}{Ejemplo}
  \newtheorem{ex}{Problema}

 \theoremstyle{remark}
 \newtheorem{nota}{Nota}


 \newcommand{\ba}{\mathscr{B}}
%%COMANDOS PARA CAMPOS
 \renewcommand{\Re}{\mathbb{R}}
 \newcommand{\Z}{\mathbb{Z}}
 \newcommand{\Fi}{\mathbb{F}}
%%COMANDOS PARA ÁLGEBRAS DE LIE
  \newcommand{\g}{\mathfrak{{g}}}
  \newcommand{\h}{\mathfrak{h}}
  \newcommand{\so}{\mathfrak{so}}
  \newcommand{\gl}{\mathfrak{gl}}
  \renewcommand{\sp}{\mathfrak{sp}}
  \renewcommand{\sl}{\mathfrak{sl}}

%%DECLARAR FUNCIONES
 \DeclareMathOperator{\Span}{span}
 \DeclareMathOperator{\Rk}{Rk}
 \DeclareMathOperator{\dd}{d}
 \DeclareMathOperator{\tr}{tr}
 \DeclareMathOperator{\ad}{ad}
 \DeclareMathOperator{\Id}{\mathbb{I}}


\DeclareMathAlphabet{\mathcalligra}{T1}{calligra}{m}{n}

\def\G{\operatornamewithlimits{%
 		\mathchoice{\vcenter{\hbox{\Large $\mathfrak{S}$}}}
 		{\vcenter{\hbox{\large $\mathfrak{S}$}}}
 		{\mathrm{G}}
 		{\mathrm{G}}}}


\begin{document}$ $\\
\textbf{\Large Problemas de \'algebra exterior}\\
\textbf{\large Andr\'es Rodr\'iguez}

  \begin{ex}
    Sean $V$ y $W$ espacios vectoriales sobre $\Fi$. Demostrar que para cada funci\'on lineal $f:V\to W$ el "pullback" $f^*: \mathscr{J}^k(W) \to \mathcal{J}^k(V)$ satisface las siguientes propiedades:
    \begin{itemize}
      \item[(i)] Para todos $S\in \mathcal{J}^m(W) $ y $T\in\mathcal{J}^k(W)$ se tiene
      $$(f\circ g)^*T = (g^*\circ f^*)T$$
      $$f^*(S\otimes T) = f^* S \otimes f^* T.$$
      y
      \item[(ii)] Para todos $\omega \in \Lambda ^m (W)$ y $\eta\in\Lambda^k(W)$
      $$f^*(\omega\wedge\eta)= f^*\omega\wedge f^*\eta.$$
    \end{itemize}
  \end{ex}
  \begin{proof}$ $\\
    \begin{itemize}
      \item[(i)] Sea $g$ una funci\'on lineal $g:U\to V$. Evaluamos $k$  vectores $v_1,\ldots,v_k\in U$ en $(f\circ g)^*T$:
                  \begin{align*}
                      (f\circ g) ^* T (v_1,\ldots,v_k) & = T((f\circ g) ^* (v_1),\ldots,(f\circ g) ^*(v_k)),\\
                      &= T(f(g(v_1)),\ldots, f(g(v_k))),\\
                      &= f^* T(g(v_1),\ldots,g(v_k)), \\
                      &= (g^*\circ f^*) T(v_1,\ldots,v_k),\\
                  \end{align*}
                  por lo tanto, $(f\circ g)^*T =(g^*\circ f^*) T$.

                  Similarmente, sean $v_1,\ldots,v_{m+k}\in V$:
                  \begin{align*}
                      f^*( S \otimes T )(v_1,\ldots,v_{m+k}) & = ( S \otimes T )(f(v_1),\ldots,f(v_{m+k})),\\
                      &= S(f(v_1),\ldots, f(v_m)) \cdot T(f(v_{m+1}, \ldots, f(v_{m+k})),\\
                      &= f^* S(v_1,\ldots,v_m)f^* T(v_{m+1},\ldots,v_{m+k}), \\
                      &= f^* S\otimes f^* T(v_{1},\ldots,v_{m+k}), ,\\
                  \end{align*}
                  por lo tanto, $f^*( S \otimes T ) =  f^* S\otimes f^* T$.
      \item[(ii)] Por la definici\'on de producto cu\~na:
                  \begin{align*}
                    f^* (\omega \wedge \eta) & = f^*\left(\frac{1}{m!k!}\sum_{\sigma \in S_{m+k}} \text{sgn}(\sigma) \sigma(\omega \otimes \eta)  \right) ,\\
                    & = \frac{1}{m!k!}\sum_{\sigma \in S_{m+k}} \text{sgn}(\sigma) \sigma( f^*\left(\omega \otimes \eta)  \right) ,\\
                    & = \frac{1}{m!k!}\sum_{\sigma \in S_{m+k}} \text{sgn}(\sigma) \sigma( f^*\omega \otimes f^* \eta)  \right),\\
                    & = f^*\omega \wedge f^*\eta.
                  \end{align*}
    \end{itemize}
  \end{proof}
  \begin{ex}
    Consideremos una funci\'on lineal $f:\Re^2 \to \Re^2$ definida por
    $$f(e_1)=-e_2,\qquad f(e_2) = e_1,$$
    donde $\{e_1,e_2\}$ es la base ortonormal para $\Re^2$. Sea $\{\phi^1,\phi^2\}$ la base dual. Calcule:
    $$f^*(\phi^1), \, f^*(\phi^2), \, f^*(\phi^1\wedge \phi^2).$$
  \end{ex}
  \begin{proof}[Soluci\'on]$ $\\
    Evaluamos en un vector $v:=v_1 e_1 + v_2 e_2 \in \Re^2$:
      \begin{align*}
        f^*(\phi^1) (v) & = \phi^1 (f(v)), \\ & = \phi^1(v_2e_1 - v_1 e_2),\\ &= v_2, \\ &= \phi^2 ( v);\\
        f^*(\phi^2) (v) & = \phi ^2(f(v)), \\ & = \phi^2(v_2e_1 - v_1 e_2),\\ & = -v_1 ,\\ &= -\phi^1 ( v);
      \end{align*}
      por lo que, $f^*(\phi^1) = \phi^2$ y $f^*(\phi^2) = - \phi^1$. Luego,
      \begin{align*}
        f^*(\phi^1\wedge \phi^2) & = f^*\phi^1 \wedge f^*\phi^2,\\ &= \phi^2\wedge(- \phi^1),\\ & = \phi^1 \wedge \phi^2.
      \end{align*}

  \end{proof}
\end{document}
